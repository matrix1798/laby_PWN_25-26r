
\documentclass[12pt, a4paper]{article}
\usepackage[utf8]{inputenc}
\usepackage[T1]{fontenc}
\usepackage[polish]{babel}
\usepackage{amsmath} % Do wzorów matematycznych
\usepackage{graphicx} % Do obrazków (tutaj jako placeholder)
\usepackage{booktabs} % Do ładnych tabel
\usepackage{tabularx} % Do tabel o ustalonej szerokości
\usepackage{geometry} % Do ustawienia marginesów
\usepackage{float} % Do wymuszenia pozycji tabel/rysunków [H]
\usepackage{siunitx} % Do ładnego formatowania jednostek (opcjonalnie, tu użyję $...$)

% Ustawienie marginesów
\geometry{
	a4paper,
	total={170mm,257mm},
	left=20mm,
	top=20mm,
}

% Definicja tytułu
\title{Sprawozdanie Laboratoryjne: Pomiary Tensometryczne}
\author{Kevin Kisiel (197866) \and Mateusz Kunerowski (137900) \\ Grupa 1A}
\date{23 października 2025}

\begin{document}
	
	\maketitle
	
	\begin{abstract}
		\noindent \textbf{Przedmiot:} Przetworniki Wielkości Nieelektrycznych \\
		\textbf{Prowadzący:} [Imię i nazwisko prowadzącego]
	\end{abstract}
	
	\hrulefill
	\vspace{1em}
	
	\section{Uzupełnienie tabel pomiarowych (Polecenie 1)}
	
	Obliczenia dla tabel zostały wykonane w oparciu o następujące założenia:
	\begin{itemize}
		\item Rezystancja nominalna (bazowa) tensometrów $R_0 = 350 \, \Omega$ (na podstawie danych z instrukcji oraz wartości rezystancji dla $\epsilon$ bliskich zeru).
		\item Względna zmiana rezystancji $\Delta R / R$ jest obliczana jako:
		$$ \frac{\Delta R}{R} = \frac{R - R_0}{R_0} = \frac{R - 350}{350} $$
		\item Na podstawie danych, dla $\epsilon = 0.001$, $R = 350.7 \, \Omega$, co daje $\Delta R/R = 0.002$. Dla $\epsilon = 0.01$, $R = 357 \, \Omega$, co daje $\Delta R/R = 0.02$. Zależność ta to $\Delta R/R = 2\epsilon$, co oznacza, że stała tensometru (współczynnik $k$) wynosi \textbf{$k=2$}.
		\item Parametry $a$ i $b$ prostej aproksymacyjnej $y = ax + b$ zostały wyznaczone metodą regresji liniowej dla danych $(x, y) = (\epsilon, U_{wyj})$.
		\item Błąd nieliniowości $\delta_{nl}$ obliczono jako:
		$$ \delta_{nl} = \frac{\max|U_{wyj, pomiar} - U_{wyj, aproksy}|}{U_{wyj, max} - U_{wyj, min}} \times 100\% $$
	\end{itemize}
	
	\subsection{1a. Ćwierćmostek, $\epsilon = 0.001-0.01$, $U_{zas} = 2.5 V$}
	
	\begin{table}[H]
		\centering
		\caption{Pomiary dla ćwierćmostka przy $U_{zas} = 2.5 V$ (mały zakres $\epsilon$).}
		\begin{tabular}{cccc}
			\toprule
			\textbf{Względne odkształcenie} & \textbf{Rezystancja dekady} & \textbf{Napięcie niezrównoważenia} & \textbf{Względna zmiana rezystancji} \\
			$\epsilon$ [-] & $R$ [$\Omega$] & $U_{wyj}$ [mV] & $\Delta R/R$ [-] \\
			\midrule
			0.001 & 350.7 & 22.327 & 0.002 \\
			0.002 & 351.4 & 23.623 & 0.004 \\
			0.003 & 352.1 & 24.868 & 0.006 \\
			0.004 & 352.8 & 26.116 & 0.008 \\
			0.005 & 353.5 & 27.356 & 0.010 \\
			0.006 & 354.2 & 28.595 & 0.012 \\
			0.007 & 354.9 & 29.836 & 0.014 \\
			0.008 & 355.6 & 31.068 & 0.016 \\
			0.009 & 356.3 & 32.302 & 0.018 \\
			0.01 & 357 & 33.523 & 0.020 \\
			\midrule
			\multicolumn{2}{l}{\textbf{Prosta aproksymacyjna $y = ax + b$}} & \multicolumn{2}{l}{$a = 1244.4$, $b = 20.97$} \\
			\multicolumn{2}{l}{\textbf{Błąd nieliniowości $U_{wyj} = f(\epsilon)$ [\%]}} & \multicolumn{2}{l}{$\approx 1.52 \%$} \\
			\bottomrule
		\end{tabular}
	\end{table}
	
	\subsection{1b. Ćwierćmostek (wpływ $U_{zas}$), $\epsilon = 0.001-0.01$, $U_{zas} = 5 V$}
	
	\begin{table}[H]
		\centering
		\caption{Pomiary dla ćwierćmostka przy $U_{zas} = 5 V$ (mały zakres $\epsilon$).}
		\begin{tabular}{cccc}
			\toprule
			$\epsilon$ [-] & $R$ [$\Omega$] & $U_{wyj}$ [mV] & $\Delta R/R$ [-] \\
			\midrule
			0.001 & 350.7 & 44.321 & 0.002 \\
			0.002 & 351.4 & 47.420 & 0.004 \\
			0.003 & 352.1 & 49.810 & 0.006 \\
			0.004 & 352.8 & 52.404 & 0.008 \\
			0.005 & 353.5 & 54.885 & 0.010 \\
			0.006 & 354.2 & 57.365 & 0.012 \\
			0.007 & 354.9 & 59.845 & 0.014 \\
			0.008 & 355.6 & 62.311 & 0.016 \\
			0.009 & 356.3 & 64.774 & 0.018 \\
			0.01 & 357 & 67.234 & 0.020 \\
			\midrule
			\multicolumn{2}{l}{\textbf{Prosta aproksymacyjna $y = ax + b$}} & \multicolumn{2}{l}{$a = 2548.8$, $b = 41.74$} \\
			\multicolumn{2}{l}{\textbf{Błąd nieliniowości $U_{wyj} = f(\epsilon)$ [\%]}} & \multicolumn{2}{l}{$\approx 0.14 \%$} \\
			\bottomrule
		\end{tabular}
	\end{table}
	
	\subsection{1c. Ćwierćmostek, $\epsilon = 0.01-0.1$, $U_{zas} = 2.5 V$}
	
	\begin{table}[H]
		\centering
		\caption{Pomiary dla ćwierćmostka przy $U_{zas} = 2.5 V$ (duży zakres $\epsilon$).}
		\begin{tabular}{cccc}
			\toprule
			$\epsilon$ [-] & $R$ [$\Omega$] & $U_{wyj}$ [mV] & $\Delta R/R$ [-] \\
			\midrule
			0.01 & 357 & 33.533 & 0.02 \\
			0.02 & 364 & 45.696 & 0.04 \\
			0.03 & 371 & 57.625 & 0.06 \\
			0.04 & 378 & 69.32 & 0.08 \\
			0.05 & 385 & 80.783 & 0.10 \\
			0.06 & 392 & 92.033 & 0.12 \\
			0.07 & 399 & 103.06 & 0.14 \\
			0.08 & 406 & 113.851 & 0.16 \\
			0.09 & 413 & 124.475 & 0.18 \\
			0.1 & 420 & 134.89 & 0.20 \\
			\midrule
			\multicolumn{2}{l}{\textbf{Prosta aproksymacyjna $y = ax + b$}} & \multicolumn{2}{l}{$a = 1126.9$, $b = 21.90$} \\
			\multicolumn{2}{l}{\textbf{Błąd nieliniowości $U_{wyj} = f(\epsilon)$ [\%]}} & \multicolumn{2}{l}{$\approx 2.50 \%$} \\
			\bottomrule
		\end{tabular}
	\end{table}
	
	\subsection{1d. Ćwierćmostek (wpływ $U_{zas}$), $\epsilon = 0.01-0.1$, $U_{zas} = 5 V$}
	
	\begin{table}[H]
		\centering
		\caption{Pomiary dla ćwierćmostka przy $U_{zas} = 5 V$ (duży zakres $\epsilon$).}
		\begin{tabular}{cccc}
			\toprule
			$\epsilon$ [-] & $R$ [$\Omega$] & $U_{wyj}$ [mV] & $\Delta R/R$ [-] \\
			\midrule
			0.01 & 357 & 67.234 & 0.02 \\
			0.02 & 364 & 91.561 & 0.04 \\
			0.03 & 371 & 115.413 & 0.06 \\
			0.04 & 378 & 138.792 & 0.08 \\
			0.05 & 385 & 161.712 & 0.10 \\
			0.06 & 392 & 184.207 & 0.12 \\
			0.07 & 399 & 206.253 & 0.14 \\
			0.08 & 406 & 227.835 & 0.16 \\
			0.09 & 413 & 249.07 & 0.18 \\
			0.1 & 420 & 269.31 & 0.20 \\
			\midrule
			\multicolumn{2}{l}{\textbf{Prosta aproksymacyjna $y = ax + b$}} & \multicolumn{2}{l}{$a = 2250.3$, $b = 44.18$} \\
			\multicolumn{2}{l}{\textbf{Błąd nieliniowości $U_{wyj} = f(\epsilon)$ [\%]}} & \multicolumn{2}{l}{$\approx 2.45 \%$} \\
			\bottomrule
		\end{tabular}
	\end{table}
	
	\newpage
	\subsection{2a. Półmostek, $\epsilon = 0.001-0.01$, $U_{zas} = 2.5 V$}
	
	\begin{table}[H]
		\centering
		\caption{Pomiary dla półmostka przy $U_{zas} = 2.5 V$ (mały zakres $\epsilon$).}
		\begin{tabular}{cccc}
			\toprule
			$\epsilon$ [-] & $R$ [$\Omega$] & $U_{wyj}$ [mV] & $\Delta R/R$ [-] \\
			\midrule
			0.001 & 350.7 & 45.367 & 0.002 \\
			0.002 & 351.4 & 47.837 & 0.004 \\
			0.003 & 352.1 & 50.337 & 0.006 \\
			0.004 & 352.8 & 52.827 & 0.008 \\
			0.005 & 353.5 & 55.227 * & 0.010 \\
			0.006 & 354.2 & 57.700 & 0.012 \\
			0.007 & 354.9 & 60.135 & 0.014 \\
			0.008 & 355.6 & 62.652 & 0.016 \\
			0.009 & 356.3 & 65.104 & 0.018 \\
			0.01 & 357 & 67.560 & 0.020 \\
			\midrule
			\multicolumn{2}{l}{\textbf{Prosta aproksymacyjna $y = ax + b$}} & \multicolumn{2}{l}{$a = 2465.7$, $b = 42.87$} \\
			\multicolumn{2}{l}{\textbf{Błąd nieliniowości $U_{wyj} = f(\epsilon)$ [\%]}} & \multicolumn{2}{l}{$\approx 0.13 \%$} \\
			\bottomrule
		\end{tabular}
		\footnotesize{* Wartość nieczytelna; przyjęto $U_{wyj} = 55.227$ mV na podstawie interpolacji.}
	\end{table}
	
	\subsection{2b. Półmostek (wpływ $U_{zas}$), $\epsilon = 0.001-0.01$, $U_{zas} = 5 V$}
	
	\begin{table}[H]
		\centering
		\caption{Pomiary dla półmostka przy $U_{zas} = 5 V$ (mały zakres $\epsilon$).}
		\begin{tabular}{cccc}
			\toprule
			$\epsilon$ [-] & $R$ [$\Omega$] & $U_{wyj}$ [mV] & $\Delta R/R$ [-] \\
			\midrule
			0.001 & 350.7 & 90.470 & 0.002 \\
			0.002 & 351.4 & 95.464 & 0.004 \\
			0.003 & 352.1 & 100.442 & 0.006 \\
			0.004 & 352.8 & 105.428 & 0.008 \\
			0.005 & 353.5 & 110.375 * & 0.010 \\
			0.006 & 354.2 & 115.333 & 0.012 \\
			0.007 & 354.9 & 120.278 & 0.014 \\
			0.008 & 355.6 & 125.21 & 0.016 \\
			0.009 & 356.3 & 130.13 & 0.018 \\
			0.01 & 357 & 135.038 & 0.020 \\
			\midrule
			\multicolumn{2}{l}{\textbf{Prosta aproksymacyjna $y = ax + b$}} & \multicolumn{2}{l}{$a = 4951.8$, $b = 85.50$} \\
			\multicolumn{2}{l}{\textbf{Błąd nieliniowości $U_{wyj} = f(\epsilon)$ [\%]}} & \multicolumn{2}{l}{$\approx 0.04 \%$} \\
			\bottomrule
		\end{tabular}
		\footnotesize{* Zapisana wartość "140.375" uznano za błąd; przyjęto $110.375$ mV.}
	\end{table}
	
	\subsection{2c. Półmostek, $\epsilon = 0.01-0.1$, $U_{zas} = 2.5 V$}
	
	\begin{table}[H]
		\centering
		\caption{Pomiary dla półmostka przy $U_{zas} = 2.5 V$ (duży zakres $\epsilon$).}
		\begin{tabular}{cccc}
			\toprule
			$\epsilon$ [-] & $R$ [$\Omega$] & $U_{wyj}$ [mV] & $\Delta R/R$ [-] \\
			\midrule
			0.01 & 357 & 67.560 & 0.02 \\
			0.02 & 364 & 91.880 & 0.04 \\
			0.03 & 371 & 115.715 & 0.06 \\
			0.04 & 378 & 139.1 & 0.08 \\
			0.05 & 385 & 162.02 & 0.10 \\
			0.06 & 392 & 184.532 & 0.12 \\
			0.07 & 399 & 206.57 & 0.14 \\
			0.08 & 406 & 228.12 & 0.16 \\
			0.09 & 413 & 249.345 & 0.18 \\
			0.1 & 420 & 270.171 & 0.20 \\
			\midrule
			\multicolumn{2}{l}{\textbf{Prosta aproksymacyjna $y = ax + b$}} & \multicolumn{2}{l}{$a = 2251.6$, $b = 45.09$} \\
			\multicolumn{2}{l}{\textbf{Błąd nieliniowości $U_{wyj} = f(\epsilon)$ [\%]}} & \multicolumn{2}{l}{$\approx 0.02 \%$} \\
			\bottomrule
		\end{tabular}
	\end{table}
	
	\subsection{2d. Półmostek (wpływ $U_{zas}$), $\epsilon = 0.01-0.1$, $U_{zas} = 5 V$}
	
	\begin{table}[H]
		\centering
		\caption{Pomiary dla półmostka przy $U_{zas} = 5 V$ (duży zakres $\epsilon$).}
		\begin{tabular}{cccc}
			\toprule
			$\epsilon$ [-] & $R$ [$\Omega$] & $U_{wyj}$ [mV] & $\Delta R/R$ [-] \\
			\midrule
			0.01 & 357 & 135.037 & 0.02 \\
			0.02 & 364 & 183.666 & 0.04 \\
			0.03 & 371 & 231.267 & 0.06 \\
			0.04 & 378 & 277.952 & 0.08 \\
			0.05 & 385 & 323.744 & 0.10 \\
			0.06 & 392 & 368.675 & 0.12 \\
			0.07 & 399 & 412.732 & 0.14 \\
			0.08 & 406 & 455.796 & 0.16 \\
			0.09 & 413 & 498.21 & 0.18 \\
			0.1 & 420 & 533.83 & 0.20 \\
			\midrule
			\multicolumn{2}{l}{\textbf{Prosta aproksymacyjna $y = ax + b$}} & \multicolumn{2}{l}{$a = 4431.5$, $b = 90.50$} \\
			\multicolumn{2}{l}{\textbf{Błąd nieliniowości $U_{wyj} = f(\epsilon)$ [\%]}} & \multicolumn{2}{l}{$\approx 0.05 \%$} \\
			\bottomrule
		\end{tabular}
	\end{table}
	
	\newpage
	\subsection{4. Wzorcowanie metodą obciążenia belki znaną siłą}
	
	Przyjęto $E_{stal} = 2.1 \times 10^4 \, \text{kG/mm}^2$.  Dane belki: $l_0 = 250$ mm, $b_0 = 60$ mm, $h = 8$ mm[cite: 268, 278, 293, 294].
	 Wzór na odkształcenie teoretyczne[cite: 348]:
	$$ \epsilon = \frac{6 l_0}{E h^2 b_0} \cdot P = \frac{6 \cdot 250}{ (2.1 \times 10^4) \cdot 8^2 \cdot 60} \cdot P \approx 1.86 \times 10^{-5} \cdot P \quad \rightarrow \quad \epsilon (10^{-6}) \approx 18.6 \cdot P $$
	Wzory na $\Delta R/R$ (pomiarowe) dla $U_{zas} = 5V = 5000 mV$:
	\begin{itemize}
		\item Półmostek: $\Delta R/R (10^{-6}) = \frac{2 \cdot \Delta U_{wyj}}{5000} \cdot 10^6 = 400 \cdot \Delta U_{wyj}$
		\item Pełen mostek: $\Delta R/R (10^{-6}) = \frac{\Delta U_{wyj}}{5000} \cdot 10^6 = 200 \cdot \Delta U_{wyj}$
	\end{itemize}
	
	\subsubsection{4a. Półmostek, $U_{zas} = 5 V$}
	 Dane pomiarowe (zmiana $U_{wyj}$ rzędu $0.454$ mV dla $P=4$ kG) [cite: 273, 289] uznano za błędne i uniemożliwiające analizę.
	
	\subsubsection{4b. Pełen mostek, $U_{zas} = 5 V$}
	
	\begin{table}[H]
		\centering
		\caption{Wzorcowanie pełnego mostka metodą obciążenia siłą.}
		\begin{tabular}{cccccc}
			\toprule
			$P$ [kG] & $U_{wyj}$ [mV] & $\epsilon (10^{-6})$ (teoret.) & $\Delta U_{wyj}$ [mV] & $\Delta R/R (10^{-6})$ (pomiar) & $k$ (pomiar) \\
			\midrule
			 0 & 1.607 [cite: 304] & 0 & 0 & 0 & - \\
			 0.5 & 1.708 [cite: 305] & 9.3 & 0.101 & 20.2 & 2.17 \\
			 1.0 & 1.808 [cite: 305] & 18.6 & 0.201 & 40.2 & 2.16 \\
			 1.5 & 1.899 [cite: 305] & 27.9 & 0.292 & 58.4 & 2.09 \\
			 2.0 & 1.998 [cite: 305] & 37.2 & 0.391 & 78.2 & 2.10 \\
			 2.5 & 2.188 [cite: 305] & 46.5 & 0.581 & 116.2 & 2.50 \\
			 3.0 & 2.379 [cite: 305] & 55.8 & 0.772 & 154.4 & 2.77 \\
			 4.0 & 2.558 [cite: 305] & 74.4 & 0.951 & 190.2 & 2.56 \\
			\midrule
			\multicolumn{4}{l}{\textbf{Prosta aproksymacyjna ($\Delta R/R = f(\epsilon)$)}} & \multicolumn{2}{l}{$a = 2.45$, $b = 3.04$} \\
			\multicolumn{4}{l}{\textbf{Błąd nieliniowości $\Delta R/R = f(\epsilon)$ [\%]}} & \multicolumn{2}{l}{$\approx 0.40 \%$} \\
			\bottomrule
		\end{tabular}
	\end{table}
	
	\newpage
	\section{Charakterystyki i analiza (Polecenia 2-7, 9, 12)}
	
	\subsection{Ćwierćmostek (Polecenia 2, 3, 4)}
	
	\subsubsection*{Polecenie 2: Charakterystyki $U_{wyj} = f(\epsilon)$ dla $\epsilon = 0.001-0.01$}
	
	\begin{figure}[H]
		\centering
		\framebox(400,250){[Wykres: Ćwierćmostek, małe odkształcenia (0.001-0.01)]}
		\caption{Charakterystyki $U_{wyj} = f(\epsilon)$ dla ćwierćmostka ($\epsilon = 0.001 \div 0.01$).  Linia górna: $U_{zas} = 5 V$ [cite: 62-74] , linia dolna: $U_{zas} = 2.5 V$[cite: 45].}
	\end{figure}
	
	Wykres przedstawia dwie linie o silnym trendzie liniowym. Linia dla $5 V$ leży wyraźnie wyżej i ma większe nachylenie niż linia dla $2.5 V$.
	
	\subsubsection*{Polecenie 3: Charakterystyki $U_{wyj} = f(\epsilon)$ dla $\epsilon = 0.01-0.1$}
	
	\begin{figure}[H]
		\centering
		\framebox(400,250){[Wykres: Ćwierćmostek, duże odkształcenia (0.01-0.1)]}
		\caption{Charakterystyki $U_{wyj} = f(\epsilon)$ dla ćwierćmostka ($\epsilon = 0.01 \div 0.1$).  Linia górna: $U_{zas} = 5 V$ [cite: 107-121] , linia dolna: $U_{zas} = 2.5 V$ [cite: 87-96].}
	\end{figure}
	
	Podobnie jak na poprzednim wykresie, linia dla $5 V$ ma około dwukrotnie większe nachylenie. Obie charakterystyki wykazują lekkie zakrzywienie (nieliniowość).
	
	\subsubsection*{Polecenie 4: Wnioski dla ćwierćmostka}
	
	\begin{itemize}
		\item \textbf{Czy napięcie zasilania $U_{zas}$ wpływa na czułość?} \\
		\textbf{Tak.} Czułość ($S = dU_{wyj}/d\epsilon$) jest wprost proporcjonalna do napięcia zasilania.
		\begin{itemize}
			\item Dla $\epsilon=0.001-0.01$: $S_{2.5V} \approx 1244$, $S_{5V} \approx 2549$. Stosunek: $2549 / 1244 \approx 2.05$.
			\item Dla $\epsilon=0.01-0.1$: $S_{2.5V} \approx 1127$, $S_{5V} \approx 2250$. Stosunek: $2250 / 1127 \approx 2.00$.
		\end{itemize}
		Podwojenie napięcia zasilania skutkuje podwojeniem czułości.  Wynika to z formuły dla ćwierćmostka $U_{wyj} \approx \frac{1}{4} \frac{\Delta R}{R} U_{zas} = \frac{1}{4} k \epsilon U_{zas}$ [cite: 601-603].
		
		\item \textbf{Czy błąd nieliniowości zależy od zakresu zmian $\epsilon$?} \\
		 \textbf{Tak.} Układ ćwierćmostka jest nieliniowy, co wynika z pełnego wzoru: $U_{wyj} = \frac{\Delta R/R}{4+2(\Delta R/R)} U_{zas}$[cite: 601].
		\begin{itemize}
			\item Dla $U_{zas} = 2.5 V$: Błąd wzrósł z $\approx 1.52 \%$ (małe $\epsilon$) do $\approx 2.50 \%$ (duże $\epsilon$).
			\item Dla $U_{zas} = 5 V$: Błąd wzrósł z $\approx 0.14 \%$ (małe $\epsilon$) do $\approx 2.45 \%$ (duże $\epsilon$).
		\end{itemize}
		Im większy zakres $\epsilon$, tym bardziej człon $2(\Delta R/R)$ w mianowniku wpływa na wynik, powodując wzrost nieliniowości.
	\end{itemize}
	
	\subsection{Półmostek (Polecenia 5, 6, 7)}
	
	\subsubsection*{Polecenie 5: Charakterystyki $U_{wyj} = f(\epsilon)$ dla $\epsilon = 0.001-0.01$}
	
	\begin{figure}[H]
		\centering
		\framebox(400,250){[Wykres: Półmostek, małe odkształcenia (0.001-0.01)]}
		\caption{Charakterystyki $U_{wyj} = f(\epsilon)$ dla półmostka ($\epsilon = 0.001 \div 0.01$).  Linia górna: $U_{zas} = 5 V$ [cite: 162-172] , linia dolna: $U_{zas} = 2.5 V$ [cite: 137-145].}
	\end{figure}
	
	Wykres przedstawia dwie linie o bardzo wysokiej liniowości. Czułość dla $5 V$ jest dwukrotnie większa niż dla $2.5 V$.
	
	\subsubsection*{Polecenie 6: Charakterystyki $U_{wyj} = f(\epsilon)$ dla $\epsilon = 0.01-0.1$}
	
	\begin{figure}[H]
		\centering
		\framebox(400,250){[Wykres: Półmostek, duże odkształcenia (0.01-0.1)]}
		\caption{Charakterystyki $U_{wyj} = f(\epsilon)$ dla półmostka ($\epsilon = 0.01 \div 0.1$).  Linia górna: $U_{zas} = 5 V$ [cite: 212-230] , linia dolna: $U_{zas} = 2.5 V$ [cite: 189-199].}
	\end{figure}
	
	Zależność pozostaje wysoce liniowa nawet w dużym zakresie $\epsilon$.
	
	\subsubsection*{Polecenie 7: Wnioski dla półmostka}
	
	\begin{itemize}
		\item \textbf{Czy napięcie zasilania $U_{zas}$ wpływa na czułość?} \\
		\textbf{Tak.} Podobnie jak w ćwierćmostku, czułość jest wprost proporcjonalna do $U_{zas}$.
		\begin{itemize}
			\item Dla $\epsilon=0.001-0.01$: $S_{2.5V} \approx 2466$, $S_{5V} \approx 4952$. Stosunek: $\approx 2.01$.
			\item Dla $\epsilon=0.01-0.1$: $S_{2.5V} \approx 2252$, $S_{5V} \approx 4432$. Stosunek: $\approx 1.97$.
		\end{itemize}
		Podwojenie napięcia zasilania podwaja czułość.
		
		\item \textbf{Czy błąd nieliniowości zależy od zakresu zmian $\epsilon$?} \\
		 \textbf{Nie (w sposób znaczący).} W układzie półmostka kompensacyjnego ($\epsilon_1 = \epsilon$, $\epsilon_2 = -\epsilon$), wzór teoretyczny $U_{wyj}=\frac{1}{2}(\frac{k\epsilon_1 - k\epsilon_2}{2+k\epsilon_1+k\epsilon_2})U_{pot}$ [cite: 635] upraszcza się, ponieważ człony nieliniowe $k\epsilon_1$ i $k\epsilon_2$ w mianowniku znoszą się. Obliczone błędy nieliniowości są bardzo małe (wszystkie $\delta_{nl} < 0.15 \%$) i nie wykazują systematycznego wzrostu wraz z zakresem $\epsilon$.
	\end{itemize}
	
	\subsection{Wzorcowanie siłą i uwagi końcowe (Polecenia 9, 10, 12)}
	
	\subsubsection*{Polecenie 9: Wzorcowanie metodą obciążenia siłą}
	
	\begin{figure}[H]
		\centering
		\framebox(400,250){[Wykres: Wzorcowanie siłą, pełen mostek (U vs Epsilon)]}
		\caption{Charakterystyka $U_{wyj} = f(\epsilon)$ dla pełnego mostka (dane z tab. 4b).}
	\end{figure}
	
	Dane dla półmostka (zad. 4a) były błędne. Porównanie teoretyczne:
	\begin{itemize}
		 \item Półmostek (1 rozciągany, 1 ściskany): $S_{pół} \propto \frac{k\epsilon}{2} U_{zas}$[cite: 635].
		 \item Pełen mostek (2 rozciągane, 2 ściskane): $S_{pełen} \propto k\epsilon \cdot U_{zas}$[cite: 665].
	\end{itemize}
	\textbf{Większą czułością charakteryzuje się układ pełnego mostka} (teoretycznie 2x większą niż półmostek i 4x większą niż ćwierćmostek).
	
	\subsubsection*{Polecenie 10: Belka o przekroju równomiernym}
	 Wykonanie tego polecenia nie jest możliwe, ponieważ protokół pomiarowy (`PWN\_Lab1\_pomiary.pdf`) nie zawiera żadnych danych pomiarowych dla tego zadania (co odpowiada tabelom C1.14 i C1.15 z instrukcji [cite: 854-863]).
	
	\subsubsection*{Polecenie 12: Komentarze i uwagi}
	\begin{enumerate}
		\item Pomiary laboratoryjne potwierdziły, że czułość mostka tensometrycznego jest wprost proporcjonalna do napięcia zasilania $U_{zas}$.
		\item Wykazano kluczową zaletę układów różnicowych (półmostek, pełen mostek) nad ćwierćmostkiem: \textbf{kompensację nieliniowości}.  Błąd nieliniowości dla ćwierćmostka rósł wraz z zakresem odkształceń, podczas gdy dla półmostka pozostawał pomijalnie mały[cite: 889, 893].
		 \item Układy półmostka i pełnego mostka oferują wyższą czułość niż ćwierćmostek[cite: 895, 897].
		\item Wystąpiły błędy pomiarowe:
		\begin{itemize}
			 \item Dane dla zadania 4a (półmostek, obciążenie siłą) [cite: 273-289] są wyraźnie błędne (niemal stałe napięcie wyjściowe).
			 \item W zadaniu 2b [cite: 169] wystąpił prawdopodobny błąd zapisu wartości pomiarowej.
			\item Obliczenia dla zadania 4b wymagały przyjęcia $E_{stal}$ z literatury, co wprowadza niepewność do wyznaczonych wartości $\epsilon$ i $k$.
		\end{itemize}
	\end{enumerate}
	
\end{document}
```